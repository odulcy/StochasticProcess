\documentclass[11pt,a4paper]{book}
\usepackage[francais]{babel}
\usepackage[utf8]{inputenc}
\usepackage{amssymb}
\usepackage{amsmath}
%\usepackage[landscape, twocolumn,left=20mm,top=15mm,bottom=15mm,right=20mm]{geometry}
\usepackage{geometry}
%\usepackage[cyr]{aeguill}
\usepackage{mathrsfs}
\usepackage{framed}
\usepackage{enumerate}
\usepackage{eurosym}
\usepackage{fancyhdr} %Faire les entêtes et bas de page
%\usepackage{indentfirst} %Pour avoir toujours une indentation au début d'un paragraphe, section...
\usepackage{listings}%pour insérer du code source
\usepackage{cancel}
\usepackage{afterpage}
\usepackage{pifont}
\usepackage{hyperref}
\usepackage{lscape}% Pour pouvoir changer en paysage avec env. landscape
\usepackage{pdflscape} %support pdfLatex
\usepackage{diagbox} %Pour faire une diago dans une cell
\usepackage{xcolor} %Insérer un peu de la couleur
\usepackage{enumitem} % Pour modifier la puce des énumérations
\usepackage{array} %Pour faire des beaux tableaux
\usepackage{nicefrac} % Pour avoir Z/nZ
\usepackage{tabularx}%Pour avoir des tableaux à dimensions réglables
\usepackage{pdfpages}%Importer des pdfs
\usepackage{ulem}
\usepackage{color}
\usepackage{tikz}
\usepackage{fancybox}
%\usepackage{sistyle}
\usepackage[nottoc, notlof, notlot]{tocbibind}
\usepackage{subcaption}
\usepackage{bbm} % Pour avoir une fonction indicatrice \mathbbm{1}
%Utilisez le package tocbibind, capable de créer des entrées pour la bibliographie, l'index et aussi la table des matières (!), les listes des figures et des tables. Ces trois derniers éléments n'étant pas du meilleur effet, on lui pourra passer les options nottoc, notlof et notlot.

%\usepackage[square,numbers,sort&compress]{natbib}
%\usepackage[titles]{tocloft}
%\setlength{\cftbeforesecskip}{3pt plus.2pt}
%\cftpagenumbersoff{section} % Pour supprimer les numéros
%\cftpagenumbersoff{subsection} % Pour supprimer les numéros
\usepackage{mdframed} %Pour faire une ligne verticale



\geometry{hmargin=1.3cm,top=1.9cm,bottom=1.9cm}

\newcommand{\der}{\mathrm{d}}
\newcommand{\dps}{\displaystyle}
\newcommand{\ptitle}[1]{\ding{113} \textsc{#1}}

\newcommand{\kbt}{k_\mathrm{B}\mathrm{T}}
\newcommand{\G}{\mathrm{G}}
\newcommand{\C}{\mathrm{C}}
\newcommand{\M}{\mathrm{M}}
\newcommand{\K}{\mathrm{K}}
\newcommand{\EI}{\mathrm{U}}
\newcommand{\Entr}{\mathrm{S}}
\renewcommand{\S}{\mathrm{S}}
\newcommand{\N}{\mathrm{N}}
\newcommand{\Z}{\mathrm{Z}}
\newcommand{\Cp}{\mathrm{Cp}}


\newcommand{\ei}{\mathrm{E}_i}
\newcommand{\zun}{\mathrm{Z}_1}


\newcommand{\R}{\mathbb{R}}
\newcommand{\Normale}{\mathcal{N}}
\newcommand{\proj}{\mathrm{P}}
\newcommand{\cov}{\mathrm{cov}}
%\newcommand{\limsup}{\mathrm{limsup}}

\renewcommand{\P}{\mathbb{P}}
\newcommand{\E}{\mathbb{E}}
\newcommand{\V}{\mathbb{V}}
\newcommand{\Vect}{\mathrm{Vect}}
\newcommand{\F}{\mathcal{F}}
\newcommand{\B}{\mathcal{B}}

\newcommand{\Q}[1]{\textbf{Question #1 : }}
\newcommand{\Rq}{\textbf{Remarque : }}
\newcommand{\Def}{\textbf{Définition : }}
\newcommand{\Prop}{\textbf{Propriété : }}
\newcommand{\Props}{\textbf{Propriétés : }}
\newcommand{\Voc}{\textbf{Vocabulaire : }}
\newcommand{\Cor}{\textbf{Corollaire : }}
\newcommand{\Th}{\textbf{Théorème : }}
\newcommand{\Preuve}{\noindent{}\textbf{Preuve : }}

\renewcommand{\leq}{\leqslant}
\renewcommand{\geq}{\geqslant}


\newtheorem{definition}{Définition}[section]
\newtheorem{prop}{Propriété}[section]
\newtheorem{voc}{section}[section]
\newtheorem{thm}{Théorème}[section]
\newtheorem{cor}{Corollaire}[thm]
\newtheorem{lem}{Lemme}[thm]
\newtheorem{ex}{Exemple}[section]
\newtheorem{exo}{Exercice}[section]
%\setlength{\parindent}{0pt}
\newcommand{\ud}{\mathrm{d}}

\newcommand{\intdr}{\int\hspace{-9.6pt}\bigcirc\hspace{-10pt}\int\hspace{-4.5pt}}
\newcommand{\oiint}{{\int \hskip -3 mm \int \hskip -5 mm \bigcirc}}

%%%%%%%%%%%%%%%%%%%%%%%%%%%%%%%%%%%%%%%%%%%%%%%%%%%%%%%%%%%%%%%%%%%%%%%%%%%%%%
\setlength{\shadowsize}{2pt}
\newcommand{\Cadre}[1]{
\begin{center}
\setlength{\fboxsep}{10pt}
\shadowbox{
\begin{minipage}{17cm}
#1
\end{minipage}
}
\end{center}}


\newcommand{\cadre}[2]{
\begin{center}
\boxput*(0,1){\colorbox{white}{\textsc{#1}}}
{
\setlength{\fboxsep}{10pt}
\shadowbox{
\begin{minipage}{13.7cm}
#2
\end{minipage}
}}
\end{center}}

\newcommand{\cadreOval}[2]{
\begin{center}
\boxput*(0,1){\colorbox{white}{\textsc{#1}}}
{
\setlength{\fboxsep}{10pt}
\Ovalbox{
\begin{minipage}{10.7cm}
#2
\end{minipage}
}}
\end{center}}

\newcommand{\ligne}{\begin{center}
	\rule{\linewidth}{.5pt}
\end{center}}
%%%%%%%%%%%%%%%%%%%%%%%%%%%%%%%%%%%%%%%%%%%%%%%%%%%%%%%%%%%%%%%%%%%%%%%%%%%%%%%%%

\newmdenv[topline=false,rightline=false]{leftbot}

%Pour renommer le sommaire
\addto\captionsfrench{% Replace "english" with the language you use
  \renewcommand{\contentsname}%
    {Sommaire}%
}


\title{Processus Stochastiques}
\author{Olivier DULCY}
\date{}

\fancypagestyle{theme}{
\fancyhead[L]{Olivier DULCY}
\fancyhead[R]{}
\fancyhead[C]{}
\fancyfoot[C]{\thepage}
\fancyfoot[R]{}
\fancyfoot[L]{Télécom SudParis 2018-2019}
\renewcommand{\footrulewidth}{1pt}
\renewcommand{\headrulewidth}{1pt}}

\fancypagestyle{garde}{
\fancyhead[L]{}
\fancyhead[C]{}
\fancyhead[R]{}
\fancyfoot[C]{}
\fancyfoot[R]{Olivier DULCY}
\fancyfoot[L]{}
\renewcommand{\footrulewidth}{0pt}
\renewcommand{\headrulewidth}{0pt}}

\pagestyle{theme}

\newcommand{\HRule}{\rule{\linewidth}{0.5mm}}
\newcommand\tikzmark[2][]{
  \tikz[remember picture,inner sep=\tabcolsep,outer sep=0,baseline=(#1.base),align=left]{\node[minimum width=\hsize](#1){$#2$};}
}

\begin{document}
\maketitle

\frontmatter

\mainmatter

\chapter{Théorème de Kolmogorov}

$\B(\R^d)$ désigne la tribu borélienne de $\R^d$. \\

\section{Rappels de probabilités}

\begin{definition}
  Un processus stochastique X est la donnée de l'ensemble défini par $\{X_t; 0\leq t < +\infty\}$, où à $t$ fixé, $X_t$ est une variable aléatoire définie sur $(\Omega,\F)$ à valeurs dans $(\R^d,\B(\R^d))$.
\end{definition}

\Rq On voit bien que $t$ peut prendre un ensemble \textbf{infini} de valeurs.
\chapter{Mouvement brownien et calcul différentiel stochastique}
\section{Mouvement brownien}

\Def Soit $X$ un mouvement brownien. On dit que le mouvement est de Markov si $\forall t_1,\ldots,t_n$, $p(x_{t_n}\vert x_{t_1}, \ldots, x_{t_n}) = p(x_n \vert x_{t_{n-1}})$.

\Prop $p(b\vert a) \sim \Normale(m_b + C^T A^{-1}(a-m_a), B - C^T A^{-1} C)$ où 
$
\begin{pmatrix}
A & C \\
C^T & B
\end{pmatrix}
$

\Q1 Montrer que $p(x_{t_n} \vert x_{t_{n-1}}, x_{t_{n-2}}) = p(x_{t_n} \vert x_{t_{n-1}})$

\Rq Cela veut dire que $p(u\vert v,w) = p(u \vert v) \Leftrightarrow p(u,w\vert v) = p(u\vert v)p(w \vert v)$

Montrons alors que $p(x_{t_n},x_{t_{n-2}} \vert x_{t_{n-1}}) = p(x_{t_n}\vert x_{t_{n-1}})p(x_{t_{n-2}} \vert x_{t_{n-1}})$ 

On remplit la matrice de covariance. Pour chaque coefficient, \og coeff = inf(indice1, indice2)\fg{}. On place les coefficients de manière \og intelligente \fg{} : on veut la matrice avec les entêtes $X_{t_{n-2}}$ et $X_{t_n}$. Ce qui donne (écrire $X_{t_{n-2}} X_{t_n} X_{t_{n-1}}$ au dessus de la matrice et sur le côté gauche): 

\[
\begin{pmatrix}
  t_{n-2} & t_{n-2} & t_{n-2} \\
  t_{n-2} & t_{n}   & t_{n-1} \\
  t_{n-2} & t_{n-1} & t_{n-1}
\end{pmatrix}
\]

Ainsi, $p(x_{t_n}, x_{t_{n-2}} \vert x_{t_{n-1}}) \sim \Normale(0,)$ (à terminer).


\subsection{Loi de l'arrivée à un point}
Considérons un point $a\in \R$ et un mouvement brownien $X(t)$ partant du point $0$. Nous allons étudier la loi de la variable aléatoire associant à chaque trajectoire l'instant de son arrivée au point $a$. Soit $\tau_a$ l'instant de la première arrivée au point $a$ de la trajectoire du processus partant du point $0$.  \\

\Rq Il y a symétrie entre les variables aléatoires $\tau_a$ et $-\tau_a$. On supposera donc $a>0$. \\

On recherche la fonction de répartition de $\tau_a$. On remarque que $[X(t) > a] \subset [\tau_a \leq t]$. En effet, une trajectoire ne peut pas dépasser $a$ sans l'avoir eu atteint.

Or, $\P(X_t > a \vert \tau_a < t) = \frac{1}{2}$ (par symétrie).

Ainsi,

\[ \P(\tau_a < t) = 2\P(X_t > a) = 2 \int_a^{+\infty} \frac{1}{\sqrt{2\pi}} e^{\frac{-u^2}{2t}} du \]

\subsection{Loi du maximum}

On cherche à connaître le comportement du maximum de la trajectoire, dans le cas du mouvement Brownien. \\
$M_{[0,t]} = \text{max}_{u \in [0,t]} X_u$

Ici, on connait la loi de $\tau_a$. Donc,

\[ \P(\text{max}_{u \in [0,t]} X_u < b) = \P(\tau_b > t) \]

\section{Intégrale et différentielle stochastiques}

Formule de Itô : Soit $X_t = \varphi(t,\psi_t)$ un processus, où $\varphi$ est une fonction de classe $\mathcal{C}^2$ allant de $\R^2$ dans $\R$, et $\psi_t$ un mouvement brownien.

On a alors :
$\ud X_t = \left(\dfrac{\partial }{\partial t}\varphi(t,\psi_t) + \dfrac{1}{2}\dfrac{\partial^2}{\partial y^2}\varphi(t,\psi_t)\right) \ud t + \dfrac{\partial}{\partial y}\varphi(t,\psi_t) \ud \psi_t $

Or, on cherche $\int \psi_t \ud \psi_t = \varphi(t,\psi_t))$, ce qui équivalent à $\dfrac{\partial \varphi}{\partial t} + \dfrac{1}{2}\dfrac{\partial^2 \varphi}{\partial y^2} = 0$ et $\dfrac{\partial \varphi}{\partial y} = y$.

On trouve $\varphi(t,y) = \frac{1}{2}(y^2-t)$

Trouver une solution de $\ud X_t = aX_t \ud t + b X_t \ud \psi_t$.

On cherche une solution de la forme $x_t = ce^{at}$. Avec la formule d'Itô, si $X_t = \varphi(t,\psi_t)$, en identifiant les parties $\ud t$ et $\ud \psi_t$, on a :

\[ \dfrac{\partial }{\partial t}\varphi(t,\psi_t) + \dfrac{1}{2}\dfrac{\partial^2}{\partial y^2}\varphi(t,\psi_t) = a\varphi(t,\psi_t) \]
et 
\[\dfrac{\partial}{\partial y}\varphi(t,\psi_t) = b\varphi(t,\psi_t) \]

On trouve :

\[ \varphi(t,\psi_t) = e^{(a-\frac{b^2}{2})t + b\psi_t} \]


Or $m_t = \E[X_t]$
\chapter{Résultats utiles}

\Th : (théorème centrale limite) : 

Soit $X_1, X_2,\ldots$ une suite de variables aléatoires réelles définies sur le même espace de probabilité, indépendantes et identiquement distribuées suivant la même loi D. Supposons que l'espérance $\mu$ et l'écart-type $\sigma$ de D existent et soient finis avec $\sigma \neq 0$.

Considérons la somme $S_n = \sum_{k=1}^n X_k$. Alors 
\begin{itemize}
  \item l'espérance de $S_n$ est $n\mu$ et
  \item l'écart-type de $S_n$ est $\sigma\sqrt{n}$
\end{itemize}

De plus, quand $n$  est \og assez grand\fg{}, la loi normale $\Normale(n\mu, n\sigma^2)$ est une bonne approximation de la loi de $S_n$.

On pose : 
\[ \Bar{X_n} = \frac{S_n}{n} \]
et 
\[ Z_n = \frac{S_n - n\mu}{\sigma\sqrt{n}} \]

La variable $Z_n$ est centrée et réduite.

Le théorème central limite énonce alors que la suite de variables aléatoires $Z_1, \ldots Z_n, \ldots$ converge en loi vers une variable aléatoire $Z$, définie sur le même espace probabilisé, et de loi normale centrée réduite $\Normale(0,1)$ lorsque $n$ tend vers l'infini.

Cela signifie que si $\Phi$ est la fonction de répartition de $\Normale(0,1)$, alors pour tout réel $z$ :

\[ \lim_{n \to \infty} \mathbb P(Z_n \le z) = \Phi(z) \]

ou, de façon équivalente :

\[ \lim_{n \to \infty}\mathbb P\left(\frac{\overline X_n - \mu}{\sigma/\sqrt n}\leq z\right) = \Phi(z)\]

\end{document}
